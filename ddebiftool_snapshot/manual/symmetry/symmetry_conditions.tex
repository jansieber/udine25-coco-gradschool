\documentclass[11pt]{scrartcl}
\usepackage[scaled=0.9]{helvet}
\usepackage[T1]{fontenc}
\usepackage[scaled=0.9]{beramono}
\usepackage{amsmath,graphicx,upquote,mathtools}
\usepackage{gensymb,paralist}
\usepackage{mathpazo,esint}
\usepackage{eulervm}
%\usepackage[notref,notcite]{showkeys}
%\usepackage[charter]{mathdesign}
\usepackage{color,listings,calc,url}
\typearea{12}
\usepackage[pdftex,colorlinks]{hyperref}
\definecolor{darkblue}{cmyk}{1,0,0,0.8}
\definecolor{darkred}{cmyk}{0,1,0,0.7}
\hypersetup{anchorcolor=black,
  citecolor=darkblue, filecolor=darkblue,
  menucolor=darkblue,pagecolor=darkblue,urlcolor=darkblue,linkcolor=darkblue}
%\renewcommand{\floor}{\operatorname{floor}}
\newcommand{\mt}[1]{\mathrm{#1}}
\newcommand{\id}{\mt{I}}
\newcommand{\matlab}{\texttt{Matlab}}
\renewcommand{\i}{\mt{i}}
\renewcommand{\d}{\mathop{}\!\mathrm{d}}
\renewcommand{\epsilon}{\varepsilon}
\renewcommand{\phi}{\varphi}
\newcommand{\sign}{\operatorname{sign}}
\newcommand{\atant}{\blist{atan2}}
\providecommand{\e}{\mt{e}}
\newcommand{\re}{\mt{Re}}
\newcommand{\im}{\mt{Im}}
\newcommand{\nc}{n_\mt{c}}
\newcommand{\ns}{n_\mt{s}}
\newcommand{\nint}{{n_\mt{int}}}
\newcommand{\Pc}{P_\mt{c}}
\newcommand{\Ps}{P_\mt{s}}
\newcommand{\psh}{p_\mt{sh}}
\newcommand{\qsh}{q_\mt{sh}}
\newcommand{\nic}{n_\mt{ic}}
\newcommand{\R}{\mathbb{R}}
\newcommand{\C}{\mathbb{C}}
\newcommand{\dint}{\sqint}
\DeclareMathOperator{\sym}{Sym}
\newcommand{\symst}{\sym_\mathrm{st}}
\newcommand{\sympo}{\sym_\mathrm{po}}
\newcommand{\warn}[1]{\textbf{\color{red}(#1)}}
\usepackage{microtype}

\definecolor{var}{rgb}{0,0.25,0.25}
\definecolor{comment}{rgb}{0,0.5,0} \definecolor{kw}{rgb}{0,0,0.5}
\definecolor{str}{rgb}{0.5,0,0}
\newcommand{\mlvar}[1]{\lstinline[keywordstyle=\color{var}]!#1!}
\newcommand{\blist}[1]{\mbox{\lstinline!#1!}}  \newlength{\tabw}
\lstset{language=Matlab,%
  basicstyle={\ttfamily\small},%
  commentstyle=\color{comment},%
  stringstyle=\color{str},%
  keywordstyle=\color{kw},%
  identifierstyle=\color{var},%
  upquote=true,%
  deletekeywords={beta,gamma,mesh}%
} \title{Description of symmetry conditions imposed by \texttt{dde\_stst\_lincond} and \texttt{dde\_psol\_lincond}} \author{Jan Sieber}\date{\today}
\begin{document}
\maketitle
\noindent 
\section{Overview}
\label{sec:quick}


\section{Functionality}
\label{sec:extra}

The standard folder \texttt{ddebiftool} contains functions that
implement extra conditions that are useful for imposing symmetry on the obtained solutions. They are applicable to
\begin{compactitem}
\item steady states (including fold and Hopf bifurcations),
\item periodic orbits (including their extended types used for period
  doubling, torus bifurcation and fold bifurcation).
\end{compactitem}
Relevant functions:
\begin{lstlisting}
function [r,J]=dde_stst_lincond(point,varargin)
function [r,J]=dde_psol_lincond(point,varargin)
\end{lstlisting}
where the latter two functions are wrappers(adapt to your own folder structure).
\subsection{\blist{dde_stst_lincond}}
\label{sec:stst}
\warn{Untested for complex Hopf case} The argument
\blist{point} is a point of type (field \blist{'kind'}) of type
\blist{'stst'}, \blist{'fold'} or \blist{'hopf'}. Let us denote
\begin{displaymath}
  n_x=\blist{size(point.x,1)}.
\end{displaymath}
The further optional
arguments are name-value pairs. Using \blist{dde_stst_lincond} one can
impose linear conditions $\symst x=0$, where $\symst x$ has the form
\begin{align}
  \label{stst:cond}
  \symst x:=\Pc\left[T\e^{\textstyle 2\pi\i p/q}-I\right]\Ps x-p_j.
\end{align}
Possible name-value pairs specifying terms in \eqref{stst:cond}:
\begin{itemize}
\item \blist{'fieldname'} (mandatory): \blist{'x'} (for points of type \blist{'stst'}, \blist{'fold'} or \blist{'hopf'}) or \blist{'v'} (for
points of type \blist{'fold'} or \blist{'hopf'}),
\item \blist{'condprojmat'}: $\Pc\in\C^{\nc\times \ns}$ or
  $\R^{\nc\times \ns}$ with default $\nc=0$, $\ns=n_x$,
\item \blist{'trafo'}: $T\in\C^{\ns\times\ns}$ or $\R^{\ns\times\ns}$ with default $T=I_{n_x}$,
\item \blist{'stateprojmat'}: $\Ps\in\C^{\ns\times n_x}$ or $\R^{\ns\times n_x}$ with default $\Ps=I_{n_x}$,
\item \blist{'rotation'}: $(p,q)\in\{0,1,\ldots\}\times\{1,2,\ldots\}$
  with default $p=0$, $q=1$. If $q=2p>0$, then $\symst x=-\Pc[T+I]\Ps x-p_j$
  is real. If $p=0$, $q>0$, then $\symst x=\Pc[T-I]\Ps x-p_j$ is real. \warn{untested for non-default}
\item \blist{'res_parameters'}: index $j$ into  \blist{'parameter'} field of \blist{point} argument with default \blist{[]} such that the term $-p_j$ in \eqref{stst:cond} is ignored. \warn{untested for non-default}
\end{itemize}
The matrices $\Pc$, $\Ps$ or $T$ can be complex only if the point type
is \blist{'hopf'} and \blist{'fieldname'} is \blist{'v'}. In this case
the complex $\symst x\in\C^{\nc}$ is split into
$(\re\symst x;\im\symst x)\in\R^{2\nc}$.
\subsection{\blist{dde_psol_lincond}}
\label{psol}
\textbf{\color{red}(Untested for complex rotations ($p\neq0$)} The argument
\blist{point} is a point of type (field \blist{'kind'}) of type
\blist{'psol'}. Let us denote
\begin{displaymath}
  n_p=\blist{size(point.profile,1)}.
\end{displaymath}
The further optional arguments are name-value pairs. Using
\blist{dde_stst_lincond} one can impose linear conditions $r=0$, where
$r$ has the form
\begin{align}
  \label{psol:cond}
  r=\sum_{\ell=1}^\nint \dint_{t_{\ell,1}}^{t_{\ell,2}}\Pc\left[TR(p/q)\Ps x(t+\psh/\qsh)-\Ps x(t)\right]\d t-p_j.
\end{align}
In \eqref{psol:cond} we use the notation
\begin{align*}
  \dint_{t_1}^{t_2}f(t)\d t=
  \begin{cases}
  \int_{t_1}^{t_2}f(t)\d t&\mbox{if $t_1\neq t_2$,}\\
  f(t_1)=f(t_2)&\mbox{if $t_1=t_2$.}
  \end{cases}
\end{align*}

Possible name-value pairs specifying terms in \eqref{stst:cond}:
\begin{itemize}
\item \blist{'fieldname'}: \blist{'profile'} (default).  If argument
  \blist{'fieldname'} is \blist{'profile'} then $n_x=n_p$ and
  $x=$\blist{point.profile}.

  Alternatives to \blist{'profile'} are \blist{'x'} or \blist{'v'} if
  the argument \blist{point} is an extended point. More precisely, if
  the point is from a periodic-orbit fold or symmetry-breaking
  continuation then profile has the format
  \blist{point.profile=cat(1,x,v)}, where \blist{x} is the solution
  profile and \blist{v} is the Floquet eigenvector. If the point is
  from a period doubling and torus bifurcation continuation then
  \blist{point.profile=cat(1,x,vre,vim)}), where \blist{x} is the
  solution profile and \blist{vre+1i*vim} is the Floquet eigenvector.
  
  If argument \blist{'fieldname'} is \blist{'x'} or \blist{'v'} then
  $n_x$ is determined by \blist{size(stateprojmat,2)}, the number of
  columns in $\Ps$. If \blist{'fieldname'} is \blist{'x'} then
  $x=$\blist{point.profile(1:nx,:)}, if \blist{'fieldname'} is
  \blist{'v'} then $x=$\blist{point.profile(end-nx+1:end,:)}.
\item \blist{'condprojmat'}: $\Pc\in\R^{\nc\times \ns}$ with default $\nc=0$, $\ns=n_x$,
\item \blist{'condprojint'}: $(t_{\ell,j}[0,1]^{\nint\times2}$ integration boundaries for integral contributions to $r$  with default $\nc=0$, $\ns=n_x$,
\item \blist{'trafo'}: $T\in\R^{\ns\times\ns}$ with default $T=I_{n_x}$,
\item \blist{'stateprojmat'}: $\Ps\in\R^{\ns\times n_x}$ with default $\Ps=I_{n_x}$,
\item \blist{'rotation'}: $(p,q)\in\{0,1,\ldots\}\times\{1,2,\ldots\}$
  with default $p=0$, $q=1$. If $q=2p>0$, then $R=-I$, if $p=0$,
  $q>0$, then $R=I$. Otherwise, $\rho=\exp(2\pi\i p/q)$ and
  \begin{align*}
    R=
    \begin{bmatrix*}[r]
      \cos\rho &-\sin\rho \\
      \sin\rho & \cos\rho 
    \end{bmatrix*}\otimes I_{\ns/2}.
  \end{align*} \warn{untested for non-default}
\item \blist{'shift'}
  $(\psh,\qsh)\in\{0,1,\ldots\}\times\{1,2,\ldots\}$ with default
  $\psh=0$, $\qsh=1$. Note that $x$ is defined on the interval
  $[0,1]$. However, $t+\psh/\qsh$ may not be inside $[0,1]$, such that
  $x(t+\psh/\qsh)$ in \eqref{psol:cond} is defined by
  \begin{align*}
    x(t+\psh/\qsh)&:=x(t_\mathrm{mod})+k_\mathrm{floor}\cdot(x(1)-x(0))\mbox{,\quad where}\\
    k_\mathrm{floor}&=\mbox{largest integer $k\leq t+\psh/\qsh$,}\\
    t_\mathrm{mod}&=t+\psh/\qsh-k_\mathrm{floor}.
  \end{align*}
  This definition ensures suitable extensions for periodic $x$ and for
  rotations $x$ (e.g., if $x(t+1)=x(t)+2\pi$ for all $t\in\R$).
\item \blist{'res_parameters'}: index $j$ into \blist{'parameter'}
  field of \blist{point} argument with default \blist{[]} such that
  the term $-p_j$ in \eqref{stst:cond} is ignored.  \warn{untested for
    non-default}
\end{itemize}
The matrices $\Pc$, $\Ps$ or $T$ are all real because the complex
Floquet eigenvectors are stored as two real vectors.


\section{List of demos}\label{sec:demos}
\begin{compactitem}
\item \textbf{\texttt{pendulum}}: Inverted pendulum subject to delayed
  feedback control \cite{SK04}. This demonstrates usage of
  \blist{dde_psol_lincond} and \blist{dde_stst_lincond} for forcing
  reflection symmetry of solutions and of eigenvectors.
\end{compactitem}

{\small\bibliographystyle{unsrt} \bibliography{../manual}
}


\end{document}
